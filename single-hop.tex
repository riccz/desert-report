The single-hop network uses a CBR source that transmits to a CBR sink
using a TDMA MAC and a Hermes PHY.

\subsection{Physical layer}
The pair of Hermes modems is modeled as an i.i.d. channel with a
probability of error that depends only on the distance between the
nodes.

The modems are assumed to work at the highest bitrate
possible, \SI{150}{kb/s}, which becomes $R = \SI{87.768}{kb/s}$ taking
into account the BCH(15,11) channel-coding, the training sequence that
preceeds each frame and the 32-bit CRC.

The maximum amount of data that can be sent in one frame
is \SI{9120}{bits} so the maximum frame size, before being
channel-coded, is \SI{9152}{bits}.

The probability of error is taken from \cite{hermes} and linearly
interpolated. In the paper it is measured at a depth of \SI{0.5}{\m}
and various distances, while transmitting full frames.
%
So, assuming that each 11-bit word is lost with probability $p_w$
independently from the others, the packet error rate for a packet
consisting of $N$ words can be computed as
\begin{equation}
  p_N = p_w^N = p_{frame}^\frac{N}{N_{frame}}
  \label{eq:pN}
\end{equation}
where $p_{frame}$ is the interpolated frame error rate and $N_{frame}
= 832$ is the number of words in a full frame, counting also the CRC.
%
{ \color{red} what about the CRC of the packet? $N = \ceil{L/11}$ in
  the code, but the CRC must also be received correctly}

\subsection{Data-link layer}
The propagation delay is very high (at \SI{50}{\m} it's already about
10 times the transmission time of an ACK) so the random access MACs
like CSMA are not efficient in this case. Using TDMA, the nodes can be
seen as being linked by two i.i.d. slotted channels.

The application layer sends $L=\SI{1000}{bytes}$ of data in each
packet, its own header is long $H_{app} = \SI{24}{bytes}$ and the
transport and network protocols add $H_{tn} = \SI{4}{bytes}$. So in
the forward channel each packet consists of $L_f = \SI{1028}{bytes}$,
while in the backward channel the ACK packets are long $L_b
= \SI{28}{bytes}$.

The packet error rates of the two channels are computed
from equation (\ref{eq:pN}):
\begin{align}
p_f &= p_{frame}^\frac{\ceil{8 L_f / 11}}{N_{frame}} \approx p_{frame}^{0.899} \\
p_b &= p_{frame}^\frac{\ceil{8 L_b / 11}}{N_{frame}} \approx p_{frame}^{0.025} .
\end{align}

Their duration are $T_f = \tau + \frac{8L_f}{R} \approx \tau
+ \SI{93.7}{\ms}$ and $T_b = \tau + \frac{8L_b}{R} \approx \tau
+ \SI{2.55}{\ms}$, where $\tau$ is the propagation delay, plus a guard
time to avoid collisions due to the clock difference between the
nodes.

\subsection{ARQ}
If the CBR window size is set to one, the ARQ behaves like a
Stop-and-Wait.
%
Using TDMA or CSMA in this case does not make any
difference, provided that the time to sense the channel and the guard
interval are comparable.
%
In both cases, the average time to succesfully transmit a packet and
receive an ACK is
\begin{align}
\E{T} &= (T_f + T_b) + \E{T}(1-p_{ok}) \\
\E{T} &= \frac{T_f + T_b}{p_{ok}} \\
\text{where} \quad p_{ok} &= (1-p_f)(1-p_b)
\end{align}
so the throughput is
\begin{equation}
S = \frac{L}{\E{T}}
\end{equation}

The probability of correct frame at \SI{50}{\m} is $p_{frame} \approx
0.883$, so $(1-p_f) \approx 0.896$, $(1-p_b) \approx 0.997$ and
$p_{ok} = 0.893$. So the throughput at 50m when S\&W is used should
be \SI{43.87}{kb/s}.
