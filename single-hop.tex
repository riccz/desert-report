\section{Single-hop acoustic network}
The first network consists of two nodes in shallow water equipped with
HERMES modems. One of them generates packets of constant length
$L=\SI{1000}{bytes}$, following a Poisson process with constant rate
$\lambda$; the other node receives the packets and acknowlegeds their
reception.

\subsection{Physical layer}
The two modems are assumed to work at the highest bitrate possible,
\SI{150}{kb/s}, which becomes $R = \SI{87.768}{kb/s}$ taking into
account the BCH(15,11) channel-coding, the training sequence that
preceeds each frame and the 32-bit CRC.
%
The maximum amount of data that can be sent in one frame
is \SI{9120}{bits} so the maximum frame size, before being
channel-coded, is \SI{9152}{bits}.

The error probility is taken from \cite{hermes}, where it is measured
between two nodes at a depth of \SI{0.5}{\m} and various distances
between \SIlist{25; 120}{\m} in a port environment. In the paper the
nodes transmit full frames so, assuming that the errors on each 11-bit
word are i.i.d., the probability of correctly receiving a packet
consisting of $N$ words can be computed as
\begin{equation}
  p_N = p_w^N = p_{\mathit{fr}}^\frac{N}{N_{\mathit{fr}}}
  \label{eq:pN}
\end{equation}
where $p_w$ and $p_{\mathit{fr}}$ are the probabilities of correct
reception of, respectively, an 11-bit word and a full frame;
$N_{\mathit{fr}} = 832$ is the number of words in a full frame,
counting also the CRC.
%
The probability $p_{\mathit{fr}}$ is linearly interpolated between the
measured values.

\subsection{Data-link layer}
The data-link layer implements the CSMA-ALOHA protocol described in
\cite{proto_issues}: each node listens for a short interval $T_l$
before transmitting; if the channel is busy it tries again as soon as
the current transmission ends. When a node has more than one packet to
send it leaves a ``time gap'' of length $T_l$ between them.
%
Since there are only two nodes the CSMA-ALOHA can avoid collisions
between the data packets and the acnowledgments, under the assumption
that the source always has some packets queued and that it uses a
sliding window. If the time $T_l$ is constant each node keeps the
channel busy while it has some packets to transmit.
%
In the case of the source node this corresponds to sending out packets
one after the other until the transmit window is full. When the last
packet is received by the sink node, it replies with a number of ACKs
equal to the number packets correclty received.

\subsection{Application layer}
The application layer implements an ARQ algorithm that work in this
way:
\begin{itemize}
\item to each generated packets is assigned an incremntal sequence
  number starting from 1;
\item the generated packets are stored in a queue, then they are sent
  one by one until the number of packets in flight is equal to the
  size of the transmit window;
\item The first packet after an idle period starts the retransmission
  timer, further packets do not reset it;
\item the sink stores the received packets in a queue and replies to
  each one with a cumulative ACK that contains the SN of the next
  packet in the received sequence, eventual duplicate packets are
  dropped but still ACKed;
\item the reordered packets are dequeued and their statistics
  collected (throughput, delay and count);
\item the source either receives an ACK or a timout occurs, if the ACK
  acknowlwdges at least one pending packet the transmit window slides
  forward and the retransmission timer is reset, otherwise (timeout of
  duplicate ACK) the packet with minimum SN still in the queue is
  retransmitted.
\end{itemize}

If the transmit and receive window sizes are both set to 1, this
strategy becomes the Stop-and-Wait ARQ. If the retransmission timeout
is chosen to be greater than the round trip time (the time between the
transmission of a packet and the reception of an ACK) there can be no
collisions, so the nodes can be seen as being linked by two
i.i.d. channels.

In this case the average throughput can be computed.

The application layer sends $L=\SI{1000}{bytes}$ of data in each
packet, its own header is long $H_{app} = \SI{24}{bytes}$ and the
transport and network protocols add $H_{tn} = \SI{4}{bytes}$. So in
the forward channel each packet consists of $L_f = \SI{1028}{bytes}$,
while in the backward channel the ACK packets are long $L_b
= \SI{28}{bytes}$.

The packet error rates of the two channels are computed
from equation (\ref{eq:pN}):
\begin{align}
  p_f &= p_{frame}^\frac{\ceil{8 L_f / 11}}{N_{frame}} \approx p_{frame}^{0.899} \\
  p_b &= p_{frame}^\frac{\ceil{8 L_b / 11}}{N_{frame}} \approx p_{frame}^{0.025} .
\end{align}

Their duration are $T_f = \tau + \frac{8L_f}{R} + T_l \approx \tau +
T_l + \SI{93.7}{\ms}$ and $T_b = \tau + \frac{8L_b}{R} +T_l \approx
\tau + T_l + \SI{2.55}{\ms}$, where $\tau \approx
\frac{d}{\SI{1500}{\m\per\s}}$ is the propagation delay and the
lsitenig time $T_l$ is negligible.

The average round trip time is
\begin{align}
  \E{T} &= (T_f + T_b) + \E{T}(1-p_{ok}) \\
  \E{T} &= \frac{T_f + T_b}{p_{ok}}
\end{align}
where $T_f$, $T_b$ are the times between the start of the transmission
and the end of the reception of, respectively, a data packet and an
ACK; the probability of receiving a response ACK is $p_{ok} =
(1-p_f)(1-p_b)$, the product of the probabilities of correct reception
of a data packet and an ACK.
%
So the throughput is
\begin{equation}
  S = \frac{L}{\E{T}}
\end{equation}

The probability of correct frame at \SI{50}{\m} is $p_{fr} \approx
0.883$, so $(1-p_f) \approx 0.896$, $(1-p_b) \approx 0.997$ and
$p_{ok} = 0.893$. So the throughput at 50m when S\&W is used should be
\SI{43.87}{kb/s}.
