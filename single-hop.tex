\section{Single-hop acoustic network}
The first network consists of two nodes in shallow water equipped with
HERMES modems. One of them generates packets of constant length
$L=\SI{1000}{bytes}$, following a Poisson process with constant rate
$\lambda$; the other node receives the packets and acknowlegeds their
reception.

\subsection{Physical layer}
The two modems are assumed to work at the highest bitrate possible,
\SI{150}{kb/s}, which becomes $R = \SI{87.768}{kb/s}$ taking into
account the BCH(15,11) channel-coding, the training sequence that
preceeds each frame and the 32-bit CRC.
%
The maximum amount of data that can be sent in one frame
is \SI{9120}{bits} so the maximum frame size, before being
channel-coded, is \SI{9152}{bits}.

The error probility is taken from \cite{hermes}, where it is measured
between two nodes at a depth of \SI{0.5}{\m} and various distances
between \SIlist{25; 120}{\m} in a port environment. In the paper the
nodes transmit full frames so, assuming that the errors on each 11-bit
word are i.i.d., the probability of correctly receiving a packet
consisting of $N$ words can be computed as
\begin{equation}
  p_N = p_w^N = p_{\mathit{fr}}^\frac{N}{N_{\mathit{fr}}}
  \label{eq:pN}
\end{equation}
where $p_w$ and $p_{\mathit{fr}}$ are the probabilities of correct
reception of, respectively, an 11-bit word and a full frame;
$N_{\mathit{fr}} = 832$ is the number of words in a full frame,
counting also the CRC.
%
The probability $p_{\mathit{fr}}$ is linearly interpolated between the
measured values.

\subsection{Data-link layer}
The data-link layer implements the CSMA-ALOHA protocol described
in \cite{proto_issues}: each node listens for a short interval $T_l$
uniformly distributed in the interval $[0.1,0.5]\si{\micro\s}$ before
transmitting; if the channel is busy it tries again as soon as the
current transmission ends.
%
Since there are only two nodes the CSMA-ALOHA can avoids collisions
between the data packets and the acnowledgments if the source always
has some packets queued.
%
In this case the source node first sends out packets one after the
other until it fills the transmit window and it must wait for either a
timeout to occur or an ACK to be received. During this time the sink
node cannot reply because it always senses the channel busy; when the
last packet is received the channel becomes idle so the sink can send
the ACKs. The acknowledgements, like the data packets, are sent one
after the other so the source 

The propagation delay is very high (at \SI{50}{\m} it's already about
10 times the transmission time of an ACK) so the random access MACs
like CSMA are not efficient in this case. Using TDMA, the nodes can be
seen as being linked by two i.i.d. slotted channels.

The application layer sends $L=\SI{1000}{bytes}$ of data in each
packet, its own header is long $H_{app} = \SI{24}{bytes}$ and the
transport and network protocols add $H_{tn} = \SI{4}{bytes}$. So in
the forward channel each packet consists of $L_f = \SI{1028}{bytes}$,
while in the backward channel the ACK packets are long $L_b
= \SI{28}{bytes}$.

The packet error rates of the two channels are computed
from equation (\ref{eq:pN}):
\begin{align}
p_f &= p_{frame}^\frac{\ceil{8 L_f / 11}}{N_{frame}} \approx p_{frame}^{0.899} \\
p_b &= p_{frame}^\frac{\ceil{8 L_b / 11}}{N_{frame}} \approx p_{frame}^{0.025} .
\end{align}

Their duration are $T_f = \tau + \frac{8L_f}{R} \approx \tau
+ \SI{93.7}{\ms}$ and $T_b = \tau + \frac{8L_b}{R} \approx \tau
+ \SI{2.55}{\ms}$, where $\tau$ is the propagation delay, plus a guard
time to avoid collisions due to the clock difference between the
nodes.

\subsection{ARQ}
If the CBR window size is set to one, the ARQ behaves like a
Stop-and-Wait.
%
Using TDMA or CSMA in this case does not make any
difference, provided that the time to sense the channel and the guard
interval are comparable.
%
In both cases, the average time to succesfully transmit a packet and
receive an ACK is
\begin{align}
\E{T} &= (T_f + T_b) + \E{T}(1-p_{ok}) \\
\E{T} &= \frac{T_f + T_b}{p_{ok}} \\
\text{where} \quad p_{ok} &= (1-p_f)(1-p_b)
\end{align}
so the throughput is
\begin{equation}
S = \frac{L}{\E{T}}
\end{equation}

The probability of correct frame at \SI{50}{\m} is $p_{frame} \approx
0.883$, so $(1-p_f) \approx 0.896$, $(1-p_b) \approx 0.997$ and
$p_{ok} = 0.893$. So the throughput at 50m when S\&W is used should
be \SI{43.87}{kb/s}.
